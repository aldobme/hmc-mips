
\section{Background and Significance}

The first aim of this project is to co-operate with the VLSI class
from Harvey Mudd College, California, to design and build a MIPS-based
microprocessor.

This microprocessor will be split into a control unit, a coprocessor and a data path. The datapath contains a fetch, decode, execute and memory stage. The Adelaide team is designing the 512kB data and instructions caches for the memory stage. The cache is direct mapped write-back. Direct mapped means that each datum from memory has a unique place it can be stored in the cache. Write-back means that data is only written back to memory when it is removed from the cache.

The microprocessor will use a MIPS R2000 instruction set architecture (ISA). The MIPS ISA is a popular RISC microprocessor architecture. R2000 has 32-bit instructions including various loads, stores, arithmetic, jumps and branches, shifts, moves and exceptions.

\section{Project Specifications}

\subsection{Final Deliverables}
\begin{enumerate}[{[D}1{]}]
\item Hardware presentation of the MIPS-based microprocessor
\item Package of testing tools and report on test results
\item Extension: Report examining low power design alternatives
\end{enumerate}

\subsection{Requirements}
\begin{enumerate}[{[R}1{]}]
\item Hardware demonstration
  \begin{itemize}
  \item Running an interactive program (possibly a Web-server or ELIZA).
  \item Robust packaging so that the demonstration is easy to move and set up.  
  \item Uses MIPS-based microprocessor.
  \end{itemize}

\item Testing tools
  \begin{itemize}
    \item Software tests of design using either fast SPICE or IRSIM, depending on availability.
    \item Documentation for software testing.
    \item Detailed methods of hardware testing.
  \end{itemize}
\item Extension: Report on low power
  \begin{itemize}
  \item Examination of MIPS based microprocessor.
  \item Development of alternative designs to achieve low power.
  \item Examination and evaluation of alternative designs
  \end{itemize}
\end{enumerate}

\subsection{Reporting Requirements}

The reporting requirements for first semester are:
\begin{description}
\item[Week 4 Project Implementation Plan:] Group plan detailing what is to be accomplished during the project, and how the group intends to do it. 
\item[Week 5 Proposal Seminar:] Group seminar presenting and explaining the intended project including requirements and what will be produced.
\item[Week 8 Critical Design Review:] Individual review of another group's project.
\item[Week 12 Project Log Book:] Individual journal including rough design calculations and notes.
\end{description}

The reporting requirements for second semester are:
\begin{description}
\item[Week 9 Final Project Report:] Group report covering the completed project and the process of producing it.
\item[Week 10 Final Project Seminars:] A seminar on the completed project.  
\item[Week 11 Project Exhibition:] An exhibition displaying the project to the public.
\end{description}
